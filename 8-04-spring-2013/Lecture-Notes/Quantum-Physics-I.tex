\documentclass[a4paper,12pt]{article}

\usepackage{amsmath,amsfonts,mathtools}
\usepackage{amsthm, amssymb}
\usepackage{graphicx}
\usepackage{hyperref}
\usepackage{enumitem}

% Personal definitions
\newcommand{\lra}{\ensuremath{\longrightarrow{}}}
\newcommand{\vect}[1]{\mathbf{#1}}
\renewcommand{\qedsymbol}{\rule{0.7em}{0.7em}}
\newcommand{\tabitem}{~~\llap{\textbullet}~~}

% Theorem commands
\newtheorem{lem}{Lemma}
\newtheorem{thm}{Theorem}
\newtheorem{defn}{Definition}

% Set spacing
\setenumerate{itemsep=1.5pt,parsep=1.5pt,topsep=0.5pt}
\setlist{itemsep=1.5pt,parsep=1.5pt,leftmargin=1pt}
\setitemize{itemsep=1.5pt,parsep=1.5pt,topsep=0.5pt}

% set 1" margins on 8.5" x 11" paper
% top left is measured from 1", 1"
\topmargin       0in
\oddsidemargin   0in
\evensidemargin  0in
\headheight      0in
\headsep         0in
\topskip         0in
\textheight      9in
\textwidth       6.5in

\begin{document}
\title{Quantum Physics I Notes}
\author{Sean Wu}
\date{\today}
\maketitle

\tableofcontents

\pagebreak

% set spacing
\setlength{\parindent}{0em}
\setlength{\parskip}{1em}

\section{Lec 1: Superposition Intuition}
\begin{itemize}
  \item Physical processes in the lab are unpredictable, nondeterminate, random
  \item Probability forced by observation

\begin{description}
  \item[Uncertainty Principle:] For incompatible properties, you cannot have an object w/ defined values for both properties at the same time
\end{description}

  \item ex. position and momentum
  \item If one property is determined, the object is in superposition of values for the other property

  \item Quantum effects negligible for large objects
  \item Quantum effects only significant for small objects w/ small energies
  \item ex. atoms, electrons, molecules
\end{itemize}

\section{Lec 2: Physical Effects explained by Quantum Mechanics but not Classical Mechanics}

\begin{enumerate}
  \item Atoms exist
  \item Randomness exists
  \item Atomic Spectra are discrete and have structure
  \item Photoelectric effect
  \item Electron Diffraction
  \item Bell's Poor Inequality
\end{enumerate}

\subsection{Atoms exist}
\begin{itemize}
  \item $e^{-}$ orbiting nucleus in Bohr atom is an accelerating charged particle and so emits light (loses energy)
  \item Thus Bohr atom doesn't work classically because it collapses as the electron spirals around nucleus while releasing energy by radiation
\end{itemize}

\subsection{Randomness exists}
\begin{itemize}
  \item Self explanatory
\end{itemize}


\subsection{Atomic Spectra}
\begin{align}
  \frac{1}{\lambda} = R \cdot \Big(\frac{1}{n_1^2} -\frac{1}{n_2^2}\Big) \text{ for } n_i \in \mathbb{Z}, n_2 > n_1
\end{align}
\begin{itemize}
  \item $R$ is the Rydberg constant which depends on the element but is independent of the emission series
  \item This eqn shows that the atomic spectra are discrete and have structure, but classical mechanics doesn't have discrete energy levels (no energy quantization)
\end{itemize}

\subsection{Photoelectric Effect}
\begin{description}
  \item[$V_0$: ] Stopping voltage req to stop $e^{-}$ from being released by photoelectric effect
  \item[$I$: ] Current generated in circuit
\end{description}

\begin{center}
  \begin{tabular}{|l|l|}
   \hline
   Prediction & Result \\
   \hline
   \tabitem More intense beam $\implies$ $e^{-}$ w/ higher KE & \tabitem  Same KE regardless of intensity\\

   \tabitem  $V_0$ $\propto$ $I$ & \tabitem $V_0$ indep of intensity\\

   \tabitem $V_0$ indep of frequnecy $\nu$ & \tabitem  $V_0$ $\propto$ $\nu$\\

   \hline
  \end{tabular}
\end{center}

\begin{itemize}
  \item Rate of $e^{-}$ release depends on intensity
  \item But for $\nu < \frac{W}{h}$ (less than critical frequency), no $e^{-}$ released regardless of intensity (not enough energy)
  \item Einstein's explanation: Light comes in chunks with defined energy $E = h\nu$
  \begin{align}
    KE = h\nu -W
  \end{align}
  where $W$ is the work required to remove the $e^{-}$
  \item Recall $E = pc$  and $c = \lambda\nu$

  \begin{align}
    \therefore{}\quad &p = \frac{h}{\lambda}
  \end{align}

  \item This implies that the discrete packets of light w/ wavelength $\lambda$ have momentum $p$ by above eqn (wave-particle duality)
\end{itemize}

\subsubsection{Waves vs Particles}
\begin{itemize}
  \item Waves can interfere with themselves (Young's Double Slit)
  \item Waves are \textbf{not localized}; particles are
  \item An interference pattern (wave) implies that \textbf{amplitudes} but intensities do not
  \item Classical particles can pass through either top or bottom slit
  \item Passing classical particles through double slit leads to 2 peaks near the 2 openings
\end{itemize}

\begin{itemize}
  \item $e^{-}$ can interfere with themselves (wave behaviour) in double slit
  \item Each $e^{-}$ takes superposition of the possible paths; We don't know if it took the top or bottom bath
  \item An $e^{-}$ is neither strictly a particle nor strictly a wave
\end{itemize}


\subsubsection{Light comes in chunks}
\begin{itemize}
  \item Light has an energy and momentum
  \begin{align}
    E &= h\nu \\
    p &= \frac{h}{\lambda}
  \end{align}
\end{itemize}

\subsection{Electron Diffraction}
\begin{itemize}
  \item Bragg's Law
  \begin{align}
    \frac{1}{\lambda} &= \frac{n}{2dsin\theta} \\
    p &= \frac{h}{\lambda}
  \end{align}
\end{itemize}

\subsection{Bell's Inequality}
\begin{itemize}
  \item For 3 binary properties A, B, C, \textbf{Bell's Inequality} states
  \begin{align}
    N(A,\overline{B}) + N(B, \overline{C}) \geq N(A, \overline{C})
  \end{align}
  where $N(X,Y)$ is the number of objects with properties X and Y
  \item $e^{-}$ have 3 binary properties (angular momentum about x-, y-, and z-axes)
  \item However, it violates Bell's Inequality
  \begin{align}
    N(\uparrow_0, \downarrow_{\theta}) + N(\uparrow_{\theta}, \downarrow_{2\theta}) \leq N(\uparrow_0, \downarrow_{2\theta})
  \end{align}
  \item Can't add probabilities classically with basic addition
\end{itemize}



\end{document}
