\documentclass[a4paper,12pt]{article}

\usepackage{amsmath,amsfonts,mathtools}
\usepackage{amsthm, amssymb}
\usepackage{graphicx}
\usepackage{hyperref}
\usepackage{enumitem}

% Personal definitions
\newcommand{\lra}{\ensuremath{\longrightarrow{}}}
\newcommand{\vect}[1]{\mathbf{#1}}
\renewcommand{\qedsymbol}{\rule{0.7em}{0.7em}}
\newcommand{\tabitem}{~~\llap{\textbullet}~~}

% Theorem commands
\newtheorem{lem}{Lemma}
\newtheorem{thm}{Theorem}
\newtheorem{defn}{Definition}

% Set spacing
\setenumerate{itemsep=1.5pt,parsep=1.5pt,topsep=0.5pt}
\setlist{itemsep=1.5pt,parsep=1.5pt,leftmargin=1pt}
\setitemize{itemsep=1.5pt,parsep=1.5pt,topsep=0.5pt}

% set 1" margins on 8.5" x 11" paper
% top left is measured from 1", 1"
\topmargin       0in
\oddsidemargin   0in
\evensidemargin  0in
\headheight      0in
\headsep         0in
\topskip         0in
\textheight      9in
\textwidth       6.5in

\begin{document}
\title{Quantum Physics I Notes}
\author{Sean Wu}
\date{\today}
\maketitle

\tableofcontents

\pagebreak

% set spacing
\setlength{\parindent}{0em}
\setlength{\parskip}{1em}

\section{Lec 1: Superposition Intuition}
\begin{itemize}
  \item Physical processes in the lab are unpredictable, nondeterminate, random
  \item Probability forced by observation

\begin{description}
  \item[Uncertainty Principle:] For incompatible properties, you cannot have an object w/ defined values for both properties at the same time
\end{description}

  \item ex. position and momentum
  \item If one property is determined, the object is in superposition of values for the other property

  \item Quantum effects negligible for large objects
  \item Quantum effects only significant for small objects w/ small energies
  \item ex. atoms, electrons, molecules
\end{itemize}

\section{Lec 2: Physical Effects explained by Quantum Mechanics but not Classical Mechanics}

\begin{enumerate}
  \item Atoms exist
  \item Randomness exists
  \item Atomic Spectra are discrete and have structure
  \item Photoelectric effect
  \item Electron Diffraction
  \item Bell's Poor Inequality
\end{enumerate}

\subsection{Atoms exist}
\begin{itemize}
  \item $e^{-}$ orbiting nucleus in Bohr atom is an accelerating charged particle and so emits light (loses energy)
  \item Thus Bohr atom doesn't work classically because it collapses as the electron spirals around nucleus while releasing energy by radiation
\end{itemize}

\subsection{Randomness exists}
\begin{itemize}
  \item Self explanatory
\end{itemize}


\subsection{Atomic Spectra}
\begin{align}
  \frac{1}{\lambda} = R \cdot \Big(\frac{1}{n_1^2} -\frac{1}{n_2^2}\Big) \text{ for } n_i \in \mathbb{Z}, n_2 > n_1
\end{align}
\begin{itemize}
  \item $R$ is the Rydberg constant which depends on the element but is independent of the emission series
  \item This eqn shows that the atomic spectra are discrete and have structure, but classical mechanics doesn't have discrete energy levels (no energy quantization)
\end{itemize}

\subsection{Photoelectric Effect}
\begin{description}
  \item[$V_0$: ] Stopping voltage req to stop $e^{-}$ from being released by photoelectric effect
  \item[$I$: ] Current generated in circuit
\end{description}

\begin{center}
  \begin{tabular}{|l|l|}
   \hline
   Prediction & Result \\
   \hline
   \tabitem More intense beam $\implies$ $e^{-}$ w/ higher KE & \tabitem  Same KE regardless of intensity\\

   \tabitem  $V_0$ $\propto$ $I$ & \tabitem $V_0$ indep of intensity\\

   \tabitem $V_0$ indep of frequnecy $\nu$ & \tabitem  $V_0$ $\propto$ $\nu$\\

   \hline
  \end{tabular}
\end{center}

\begin{itemize}
  \item Rate of $e^{-}$ release depends on intensity
  \item But for $\nu < \frac{W}{h}$ (less than critical frequency), no $e^{-}$ released regardless of intensity (not enough energy)
  \item Einstein's explanation: Light comes in chunks with defined energy $E = h\nu$
  \begin{align}
    KE = h\nu -W
  \end{align}
  where $W$ is the work required to remove the $e^{-}$
  \item Recall $E = pc$  and $c = \lambda\nu$

  \begin{align}
    \therefore{}\quad &p = \frac{h}{\lambda}
  \end{align}

  \item This implies that the discrete packets of light w/ wavelength $\lambda$ have momentum $p$ by above eqn (wave-particle duality)
\end{itemize}

\subsubsection{Waves vs Particles}
\begin{itemize}
  \item Waves can interfere with themselves (Young's Double Slit)
  \item Waves are \textbf{not localized}; particles are
  \item An interference pattern (wave) implies that \textbf{amplitudes} but intensities do not
  \item Classical particles can pass through either top or bottom slit
  \item Passing classical particles through double slit leads to 2 peaks near the 2 openings
\end{itemize}

\begin{itemize}
  \item $e^{-}$ can interfere with themselves (wave behaviour) in double slit
  \item Each $e^{-}$ takes superposition of the possible paths; We don't know if it took the top or bottom bath
  \item An $e^{-}$ is neither strictly a particle nor strictly a wave
\end{itemize}


\subsubsection{Light comes in chunks}
\begin{itemize}
  \item Light has an energy and momentum
  \begin{align}
    E &= h\nu \\
    p &= \frac{h}{\lambda}
  \end{align}
\end{itemize}

\subsection{Electron Diffraction}
\begin{itemize}
  \item Bragg's Law
  \begin{align}
    \frac{1}{\lambda} &= \frac{n}{2dsin\theta} \\
    p &= \frac{h}{\lambda}
  \end{align}
\end{itemize}

\subsection{Bell's Inequality}
\begin{itemize}
  \item For 3 binary properties A, B, C, \textbf{Bell's Inequality} states
  \begin{align}
    N(A,\overline{B}) + N(B, \overline{C}) \geq N(A, \overline{C})
  \end{align}
  where $N(X,Y)$ is the number of objects with properties X and Y
  \item $e^{-}$ have 3 binary properties (angular momentum about x-, y-, and z-axes)
  \item However, it violates Bell's Inequality
  \begin{align}
    N(\uparrow_0, \downarrow_{\theta}) + N(\uparrow_{\theta}, \downarrow_{2\theta}) \leq N(\uparrow_0, \downarrow_{2\theta})
  \end{align}
  \item Can't add probabilities classically with basic addition
\end{itemize}

\section{Lec 3: The Wave Function}

\subsection{de Broglie Relations}
\begin{equation}
  \begin{aligned}
    E &\sim \hbar & \quad\quad &E = h\nu \\
    p &= \hbar k  & \quad\quad &p = \frac{h}{\lambda}
  \end{aligned}
\end{equation}

\begin{align}
  \hbar = \frac{h}{2\pi} \quad\quad  \omega = 2\pi\nu \quad\quad k = \frac{2\pi}{\lambda}
\end{align}

\begin{description}
  \item[$\omega$ :] angular frequency [rad/s]
  \item[$k$ :] wavenumber [rad/m]
\end{description}


\subsection{Systems in Classical Mechanics vs Quantum Mechanics}
\begin{itemize}
  \item In \textbf{Classical Mechanics}, an object's state is fully defined by its position and momentum vectors $\{\vect{x}, \vect{p}\}$
  \item All other properties can be found using $\vect{x}$ and $\vect{p}$
  \item ex. energy $E(\vect{x}, \vect{p})$ and angular momentum $\vect{L}(\vect{x}, \vect{p})$
  \item But in \textbf{Quantum Mechanics} (real life), there is uncertainty (Uncertainty Principle)
  \begin{align}
    \Delta \vect{x} \Delta \vect{p} \gtrsim \hbar
  \end{align}
\end{itemize}

\subsection{Quantum Mechanics Postulates}
\begin{enumerate}
  \item The state of a quantum object is \textbf{completely} specified by a wavefunction $\Psi(x)$
  \item $\mathbb{P} = |\Psi(x)|^2$ determines the probability density that the object in state $\Psi(x)$ will be found at $x$
  \begin{itemize}
    \item i.e. probability that upon measurement, the object is found at position $x$
  \end{itemize}
  \item Given two possible wavefunctions (or states) of a quantum system corresponding to distinct wavefunctions $\Psi_1(x)$ and $\Psi_2(x)$, the system can \textbf{also} be in a \textbf{superposition} of $\Psi_1(x)$ and $\Psi_2(x)$
  \begin{align}
    \Psi(x) = \alpha\Psi_1(x) + \beta\Psi_2(x) \quad \alpha, \beta \in \mathbb{C}
  \end{align}
  such that $\Psi(x)$ is properly normalized
  \begin{itemize}
    \item Wavefunction can be expressed as a linear combination of 2 possible wavefunctions (superposition)
    \item i.e. superposition of 2 quantum states results in another valid quantum state
  \end{itemize}
\end{enumerate}

\subsection{The Wavefunction $\Psi(x)$}
\begin{itemize}
  \item Wavefunction $\Psi(x)$ is a complex function and \textbf{must} be single valued and continuous
  \item The probability $|\Psi(x)|^2$ is always real and nonnegative
  \item Probability density means

  \begin{align}
    \mathbb{P}(x, x+dx) = \mathbb{P}(x)dx = |\Psi(x)|^2dx
  \end{align}

  \item Units of wavefunction are $[\Psi(x)] = \frac{1}{\sqrt{L}}$
  \item Recall that for complex numbers,

  \begin{align}
    |\beta|^2 = \beta^* \beta \\
    |e^{i\alpha}|^2 = e^{i\alpha}e^{-i\alpha} = 1
  \end{align}

\subsection{Normalization of Wavefunctions}

  \item Probability must be \textbf{normalized} such that the sum of probabilities is 1 over an interval

  \begin{align}
    \int_{All} \mathbb{P}(x)dx = \int_{All} |\Psi(x)|^2dx = 1
  \end{align}

  \item If wave function is not normalized, then use

  \begin{align}
    \mathbb{P}(x) = \frac{|\Psi(x)|^2}{\int_{All} |\Psi(x)|^2dx}
  \end{align}

\end{itemize}

\subsection{Plane Waves}
\begin{itemize}
  \item de Broglie says a particle with energy $E \sim \hbar \omega$ and momentum $p = \hbar k$ has a plane wave wavefunction
  \item General plane wave:

  \begin{align}
    \Psi(x) = e^{i(kx-wt)}
  \end{align}
  \item Note: plane wave is a complex function (need to remember that there is an imaginary component)
  \item But not all wavefunctions are plane waves; some are well localized
\end{itemize}

\subsection{Superposition of 2 waves}
\begin{itemize}
  \item Using the rule $|\beta|^2 = \beta^* \beta$, the probability with superposition is:
  \begin{align}
    \mathbb{P} &= |\alpha\Psi_1 + \beta\Psi_2|^2 = (\alpha^*\Psi_1^* + \beta^*\Psi_2^*)(\alpha\Psi_1 + \beta\Psi_2) \\ &=|\alpha|^2|\Psi_1|^2 + |\beta|^2|\Psi_2|^2 + \alpha^*\Psi_1^* \beta\Psi_2 + \alpha\Psi_1 \beta^*\Psi_2^* \\
    &= \mathbb{P}_1 + \mathbb{P}_2 + \alpha^*\Psi_1^* \beta\Psi_2 + \alpha\Psi_1 \beta^*\Psi_2^*
  \end{align}
  where $\alpha^*\Psi_1^* \beta\Psi_2 + \alpha\Psi_1 \beta^*\Psi_2^*$ are the \textbf{interference terms}

  \item The first term in the interference terms is the conjugate of the 2\textsuperscript{nd} term
  \item Therefore the interference term is real but not necessarily nonnegative, so the overall probability will still be real
  \item Note: Superposition principle and interpretation of probability as $|\Psi(x)|^2$ gives correction to classical probability (the interference terms)
  \item Shows that probabilities don't add as they do classically in Bell's Inequality (i.e. probability of both is \textbf{not} the sum of the individual probabilities)
\end{itemize}

\textbf{Wavefunctions add; probabilities do not}

\subsection{Superposition of many waves}
\begin{itemize}
  \item As you add more plane waves to a superposition, the wavefunction and probability distribution become more localized (i.e. $\Delta x$ decreases)
  \item For lots of plane waves, get a very narrow probability distribution and wavefunction
  \item There will be very few peaks, so particle is very likely to be found at those positions: $\Delta x \sim \text{small}$
  \item But that requires a superposition of many momenta (each plane wave has a different $\lambda$ and $p = \frac{h}{\lambda}$) so $\Delta p \sim \text{large}$ as required by the Uncertainty Principle

\end{itemize}

\subsubsection{Fourier Transforms for wave functions}
\begin{thm}
  Any well behaved $f(x)$ can be built by superimposing enough plane waves $e^{ikx}$

  \begin{align}
    f(x) = \frac{1}{\sqrt{2\pi}} \int^{\infty}_{-\infty}\widetilde{f}(x) e^{ikx} dk
  \end{align}
\end{thm}
where $\widetilde{f}(x)$ gives the amplitude of the plane wave with wavelength $\lambda = \frac{2\pi}{k}$

\begin{itemize}
  \item Every mode has a definite wavelength $\lambda = \frac{2\pi}{k}$
  \item Note: Fourier Transform coefficients of $x$ are all equivalent
  \item Can use Inverse Fourier Transform to get $\widetilde{f}(x)$ from $f(x)$
  \begin{align}
    \widetilde{f}(k) = \frac{1}{2\pi} \int^{\infty}_{-\infty}f(x) e^{-ikx} dx
  \end{align}
\end{itemize}

\begin{itemize}
  \item Physics Version: any $\Psi(x)$ can be expressed as the superposition of states with definite momentum $p = \hbar k$
  \begin{align}
    \Psi(x) = \frac{1}{\sqrt{2\pi}} \int^{\infty}_{-\infty}\widetilde{\Psi}(x) e^{ikx} dk
  \end{align}
  \item The Fourier Transform associates a magnitude and phase for each possible wave vector
  \item Note: if the wavefunction is well localized to a position, the Fourier Transform is not well localized ($\therefore$ not having definite momentum)
  \item Similarly, if there is definite momentum, position is not well defined, but the Fourier Transform will have a single peak (position very localized)

\end{itemize}



\subsection{Wavefunction Examples}

\section{Aside: More about Fourier Transforms}






\end{document}
