\section{Sets and Functions}
\subsection{Sets}
\begin{description}
	\item[set] A set is a collection defined by its elements. Synonymous with class or group.
\end{description}
\begin{itemize}
	\item Members of a set are not required to be related
\end{itemize}

\begin{axiom}[Axiom of Extensionality]
	For any sets $A \text{ and } B, A=B \iff A \text{ and } B$ have exactly the same
	elements
	\label{axiom:AoE}
\end{axiom}

\begin{itemize}
	\item Can name sets by
	      \begin{enumerate}
		      \item listing elements
		            \begin{itemize}
			            \item By \nameref{axiom:AoE}, when naming sets by listing elements, order and
			                  repetition don't matter
		            \end{itemize}
		      \item Naming some feature all of its elements share that nothing
		            else has
		            \begin{itemize}
			            \item Ex. $\{x| x \text{ is an inner planet}\}$
		            \end{itemize}
	      \end{enumerate}
\end{itemize}

\begin{definition}[Set Membership $\in$]
	The symbol $\in$ indicates set membership. For any $a,a \in \{x| x \text{ is }
		\rule{1cm}{0.15mm}\} \iff a \text{ is } \rule{1cm}{0.15mm}$
	\begin{itemize}
		\item E.x. $\text{Mercury} \in \{x | x \text{ is an inner planet} \}
			      \iff \text{ Mercury is an inner planet}$
	\end{itemize}
\end{definition}

\begin{description}
	\item[naive comprehension principle] Given any condition expressible by a
		formula $\phi(x)$, it is possible to form the set of all sets meeting
		that condition, denoted by $\{x| \phi(x)\}$
		\begin{itemize}
			\item n.b. naive comprehension principle cannot be upheld with full
			      generality
		\end{itemize}
	\item[Russel's paradox] naive comprehension principle allows formation
		of a \textit{set of all non-self-membered sets}, $\{x| x \not\in x\}$
\end{description}
\begin{align*}
	\{x| x \text{ is a set that isn't an element of itself}\} \in
	\{x| x \text{ is a set that isn't an element of itself}\} \\
	\iff \{x|
	x \text{
		is a set that isn't an element of itself}\}
\end{align*}
\begin{itemize}
	\item A central challenge of $20^{th}$ century was to restrict the naive
	      comprehension principle so as to avoid Russell's paradox while meeting
	      math's needs
\end{itemize}

\subsection{Set Operations: Subset, Union, Intersection, Difference}
\begin{definition}[Subset $\subseteq$]
	For any sets $A$ and $B$, $A$ is a \textit{subset} of B $\iff$ its elements are
	all elements of B. I.e. $A \subseteq B \iff A = \{x | x \in B\}$
\end{definition}
\textbf{Subset properties}
\begin{enumerate}
	\item If $A \subseteq B$ and $B \subseteq C$, then $A \subseteq C$
	\item $A \subseteq A$
	\item $A=B \iff A \subseteq B$ and $B \subseteq A$
\end{enumerate}

\begin{definition}[Null set (empty set) $\emptyset$]
	The \textit{null set} or \textit{empty set}, $\emptyset$, is the
	set that has no elements. $\emptyset \subseteq A$ for every set $A$
\end{definition}

\begin{definition}[Union of sets $\cup$]
	For any sets $A$ and $B$, the \textit{union} of $A$ and $B$ is $A \cup B =
		\{x| x \in A \text{ or } x \in B \text{ or both}\}$
\end{definition}

\textbf{Union properties}
\begin{enumerate}
	\item $A \cup B = B \cup A$
	\item $(A \cup B) \cup C = A \cup (B \cup C)$
	\item $A \cup A = A$
	\item $A \subseteq B \iff A \cup B  = B$
\end{enumerate}

\begin{definition}[Intersection of sets $\cap$]
	For any sets $A$ and $B$, the \textit{intersection} of $A$ and $B$ is defined as $A
		\cap B = \{x| x \in A \text{ and } x \not\in B\}$
\end{definition}

\textbf{Intersection properties}
\begin{enumerate}
	\item $A \cap B = B \cap A$
	\item $(A \cap B) \cap C = A \cap (B \cap C)$
	\item $A \cap A$
	\item $A \subseteq B \iff A \cap B = A$
	\item $(A \cap B) \cup C = (A \cup C) \cap (B \cup C)$
	\item $(A \cup B) \cap C = (A \cap C) \cup (B \cap C)$
\end{enumerate}

\begin{definition}[Difference of sets $\setminus$]
	For any sets $A$ and $B$, the \textit{difference} of $A$ and $B$ is defined
	as $A \setminus B = \{x | x \in A
		\text{ and } x \in B\}$
\end{definition}

\textbf{Difference Properties}
\begin{enumerate}
	\item $A \setminus A = \emptyset$
	\item $(A \cup B) \setminus C = (A \setminus C) \cup (B \setminus C)$
	\item $A \setminus (B \cup C) = (A \setminus B) \cap (A \setminus C)$
	\item $(A \cap B) \setminus C = (A \setminus C) \cap (B \setminus C)$
	\item $A \setminus (B \cap C) = (A \setminus B) \cup (A \setminus C)$
	\item $(A \setminus B) \setminus C = A \setminus (B \cup C)$
	\item $A \setminus (B \setminus C) = (A \setminus B) \cup (A \cap (B \cap
		      C))$
	\item $(A \setminus B) \cup (A \cap B) = A$
	\item $(A \setminus B) \cap (A \cap B) = \emptyset$
	\item $(A \setminus B) \cup (B \setminus A) = (A \cup B) \setminus (A
		      \cap B)$
	\item $(A \setminus B) \cap (B \setminus A) = \emptyset$
\end{enumerate}

\begin{definition}[Union of set of sets $\bigcup$]
	If $\mathcal{F}$ is a set of sets, the \textit{union} of
	$\mathcal{F}$ is $\bigcup \mathcal{F} = \{x | x \text{ is an element of at
			least  one element of } \mathcal{F}\}$
\end{definition}
\begin{itemize}
	\item $\therefore A \cup B = \bigcup \{A, B\}$
\end{itemize}

\begin{definition}[Intersection of set of sets $\bigcap$]
	If $\mathcal{F}$ is a nonempty set of sets, the \textit{intersection} of
	$\mathcal{F}$ is $\bigcap \mathcal{F} = \{ x | x \text{ is an element of every
			element of } \mathcal{F}\}$
\end{definition}
\begin{itemize}
	\item $\therefore A \cap B = \bigcap \{A,B\}$
	\item We have for a $\mathcal{F}$ a nonempty set of sets and $C$ a set
\end{itemize}
\begin{enumerate}
	\item $(\bigcup \mathcal{F}) \setminus C = \bigcup \{A \setminus C | A \in
		      \mathcal{F}\}$
	\item $(\bigcap \mathcal{F}) \setminus C = \bigcap \{A \setminus C | A \in
		      \mathcal{F}\}$
	\item $C \setminus (\bigcup \mathcal{F}) = \bigcap \{ C \setminus A | A \in
		      \mathcal{F}\}$
	\item $C \setminus(\bigcap \mathcal{F}) = \bigcup \{ C \setminus A | A \in
		      \mathcal{F}\}$
\end{enumerate}

\subsection{Ordered Pairs}
\begin{definition}[Unordered pair $\{a,b\}$]
	For given individuals $a$ and $b$, the set $\{a, b\}$ is the \textit{unordered
		pair} formed by $a$ and $b$.
\end{definition}
\begin{itemize}
	\item Called unordered because $\{a,b\} = \{b,a\}$
\end{itemize}

\begin{definition}[Ordered pair $\langle a,b\rangle $]
	For given $a$ and $b$, we form the ordered pair $\langle a,b\rangle $ so that (unless
	$a=b$ ), $\langle a,b\rangle  \neq \langle b,a\rangle $
\end{definition}

\begin{theorem}[Law of Ordered Pairs]
	For any $a,b,c,d$ $\langle a,b\rangle  = \langle c,d\rangle  \iff a=c \text{ and } b=d$
	\label{thm:LOP}
\end{theorem}
\begin{itemize}
	\item Notion of ordered pairs useful in geometry where we associate a point $P$
	      with an ordered pair $\langle x,y\rangle $
	\item Set theorists sometimes define ordered pairs as $\langle a,b\rangle  =
		      \{\{a\},\{a,b\}\}$
	      \begin{itemize}
		      \item Allows \nameref{thm:LOP} to be derived from
		            \nameref{axiom:AoE}
	      \end{itemize}
	\item Ordered triple $\langle a,b,c\rangle  = \langle \langle a,b\rangle ,c\rangle $
	\item Ordered quadruple $\langle a,b,c,d \rangle = \langle \langle a,b,c\rangle ,d\rangle  = \langle \langle \langle a,b\rangle ,c\rangle ,d\rangle $
	\item Define 1-tuple as $\langle a\rangle  = a$ for convenience
\end{itemize}

\begin{definition}[Ordered $n+1$ tuple]
	The ordered $n+1$ tuple $\langle a_1, a_2, \dots, a_n, a_{n+1} \rangle$ is the ordered
	pair where the 1st element is the ordered $n$ tuple $\langle a_1, a_2, \dots
		a_n \rangle$
	and the 2nd element is $a_{n+1}$
\end{definition}

\subsection{Functions}
\begin{definition}[Cartesian product $A \times B$]
	For sets $A \text{ and } B$, the \textit{Cartesian product} $A \times B$ is
	$A \times B = \{\langle a,b \rangle | a \in A \text{ and } b \in B\}$
\end{definition}
\begin{itemize}
	\item n.b. Cartesian product of 2 sets is a set of ordered pairs
	\item Cartesian product forms a table (set of rows times a set of
	      columns) where each cell contains ordered pairs of the form $\langle
		      \text{row value}, \text{column value} \rangle$
\end{itemize}

\begin{definition}[Function $f: A \to B$]
	A subset $f$ of $A \times B$ is said to be a \textit{function} from $A$ to
	$B$ (symbolically, $f: A \to B$). For each element $a$ of $A$, there is exactly
	1 element $b$ of $B$ with $\langle a,b \rangle \in f$
\end{definition}
\begin{itemize}
	\item n.b function is a set of ordered pairs
	\item n.b. output of $f$ uniquely determined by input
	\item if $f: A \to B$ and
	      $\langle a,b \rangle \in f$, we write $f(a) = b$
\end{itemize}

\begin{definition}[Function composition $g \circ f = g(f(x))$]
	If $f:A \to B$ and $g: B \to C$, the \textit{composition} $g \circ f$ is the
	function $A \to C$ given by $g \circ f(a) = g(f(a))$. If $h: C \to D$, we have
	$(h \circ g) \circ f = h \circ (g \circ f)$
\end{definition}
\begin{itemize}
	\item n.b. Read composition from inside to outside; for $g \circ f(x) =
		      g(f(x))$, the function $f$ is applied first
\end{itemize}

\begin{definition}[Injective (one-one)]
	A function $f: A \to: B$ is \textit{one-one} or \textit{injective} whenever
	we have $f(x) = f(y)$, we have $x=y$. I.e. never get same output for 2
	different inputs.
\end{definition}

\begin{definition}[Domain $\mathcal{D}$ and Range $\mathcal{R}$]
	If $f$ is a function from $A$ to $B$, then the set of inputs is the \textit{domain}
	of $f$, $\mathcal{D}$, and the \textit{range} of $f$ is the set of all $b \in B$
	such that for some $a \in A, f(a) = b$ (i.e. the range of $f$ is the set of
	outputs).
\end{definition}

\begin{definition}[Surjective (onto)]
	If the range of $f$ is all of $B$, $f$ is \textit{onto} or
	\textit{surjective}. I.e. function can reach all of $B$.
\end{definition}

\begin{definition}[Bijective (one-to-one correspondence)]
	If $f$ is both injective and surjective, it is \textit{bijective} or a
	\textit{one-one correspondence}
\end{definition}

\begin{definition}[Inverse $f^{-1}$]
	If $f$ is a bijection: $A \to B$, the \textit{inverse} $f^{-1}$ is
	the bijection $B \to A$ given by $f^{-1} = \{ \langle b,a \rangle | \langle
		a,b \rangle \in f\}$
\end{definition}
\begin{definition}[Identity map $ID_A$]
	The \textit{identity map} on $A$, $ID_A = \{\langle a,a, \rangle
		| a \in A\}$ is a bijection from $A$ to itself.
	We also have $f^{-1} \circ f = ID_A$ and $f \circ f^{-1} = ID_B$.
\end{definition}

\subsubsection{Examples of Functions in Set Notation}
\begin{itemize}
	\item Where $\mathbb{R}$ is the set of real numbers, $\{\langle x, x^3
		      \rangle\}$ is a bijection $\mathbb{R} \to \mathbb{R}$. The function
	      $\{\langle x, x^3 - 6x^2 + 11x - 6 \rangle | x \in \mathbb{R}\}$ is
	      \textit{surjective} but isn't \textit{one-one} because inputs 1,2,3
	      all give output 0.
	\item Arctangent function $\{\langle x,y \rangle | -\frac{\pi}{2} < y <
		      \frac{\pi}{2} \text{ and } x = \tan(y)\}$ is \textit{injective} but
	      not \textit{surjective} because its an increasing function whose range
	      is the set of numbers between $-\frac{\pi}{2}$ and $\frac{\pi}{2}$
	\item Function $\{\langle x,x^2 \rangle | x \in \mathbb{R}\}$ is neither
	      \textit{injective} nor \textit{surjective}. Not \textit{injective}
	      because gives output 4 for both inputs 2, -2. Not \textit{surjective}
	      because none of the negative numbers are in its range.
	\item Cartesian products and functions for $B = \{a,b\}$ and $G = \{x,y,z\}$
	      \[
		      B \times B = \{ \langle a,a \rangle, \langle a,b \rangle, \langle
		      b,a \rangle, \langle b,b \rangle \} \text{ is a 4 element set} .\]
	      \[
		      B \times G = \{ \langle a,x \rangle, \langle a,y \rangle, \langle
		      a,z \rangle, \langle b,x \rangle, \langle b,y \rangle, \langle b,z
		      \rangle \} \text{ is a 6 element set}
		      .\]
	      \[
		      G \times B = \{ \langle x,a \rangle, \langle x,b \rangle, \langle
		      y,a \rangle, \langle y,b \rangle, \langle z,a \rangle, \langle z,b
		      \rangle \} \text{ is a 6 element set}
		      .\]
	      \[
		      G \times G = \{ \langle x,x \rangle, \langle x,y \rangle, \langle
		      x,z \rangle, \langle y,x \rangle, \langle y,y \rangle, \langle y,z
		      \rangle, \langle z,x \rangle, \langle z,y \rangle, \langle z,z
		      \rangle\} \text{ is a 6 element set}
		      .\]
	      4 functions from $B$ to $B$
	      \begin{align*}
		      \{ \langle a,a \rangle, \langle b, a \rangle \} & \; \text{ not injective nor
		      surjective}                                                                   \\
		      \{ \langle a,a \rangle, \langle b,b \rangle \}  & \; \text{
		      bijective}                                                                    \\
		      \{ \langle a,b \rangle, \langle b,a \rangle \}  & \; \text{
		      bijective}                                                                    \\
		      \{ \langle a,b \rangle, \langle b,b \rangle \}  & \; \text{ not
			      injective nor surjective}
	      \end{align*}
\end{itemize}
