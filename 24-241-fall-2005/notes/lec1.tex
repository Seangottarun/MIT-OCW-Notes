\section{Introduction: Place of Logic Among Sciences}
\subsection{Argument}
\begin{description}
	\item[logic] science of correct argument
	\item[premise] set of accepted/assumed propositions; foundation that argument is
		based on
	\item[argument] leads from premises to conclusion
\end{description}

\begin{itemize}
	\item ex. Prosecutor tries convincing jury from premises that events
	      occurred to the guilty conclusion
\end{itemize}

\begin{description}
	\item[correct argument] anyone who accepts premises ought to accept
		conclusion
\end{description}

\begin{itemize}
	\item Correct argument $\neq$ persuasive argument
	\item ex. Appeal to prejudices (persuasive but incorrect) or correct but
	      boring (correct but not persuasive)
\end{itemize}

\begin{description}
	\item[argument correctness test]
\end{description}
\begin{itemize}
	\item Don't look at whether good reason to accept premises
	\item Just check if premises $\implies$ conclusion
	\item If argument $\implies$ conclusion, but person won't accept premise,
	      they won't accept the conclusion
	      \begin{itemize}
		      \item This is the premises' fault; not the argument's
	      \end{itemize}
	\item Argument correct $\implies$ at least as confident in conclusion
	      as in premises
\end{itemize}

\begin{description}
	\item[argument's purpose]
\end{description}
\begin{enumerate}
	\item Persuade others (ex. Prosecutor and jury)
	\item Persuade yourself (ex. Geometry proof)
\end{enumerate}

\subsection{History of Logic}
\subsubsection{Ancient Greek Geometry}
\begin{enumdescription}
	\item[axioms/basic postulates] obvious, self-evident, no demonstration
	needed
	\item[theorems] derived from axioms
\end{enumdescription}
\begin{itemize}
	\item Axioms expressed a lot of info compactly
	\item Any geometrical problem solvable by deriving answer from axioms
\end{itemize}

\subsubsection{Syllogisms [Aristotle] --- Start of Science of Logic}
\begin{itemize}
	\item Aristotle's observations
	      \begin{enumerate}
		      \item All sciences can be built from axioms to theorems
		      \item Argumentative principles for deriving theorems from axioms
		            are the same in all sciences
	      \end{enumerate}
	\item Aristotle focused on syllogisms
\end{itemize}
\begin{description}
	\item[syllogism] form of reasoning where a conclusion is drawn (valid or
		not) from two given or assumed propositions (premises). Each proposition
		shares a term with the conclusion and shares a common/middle term not in
		conclusion.
		\begin{itemize}
			\item ex. All trout are fish. All fish swim. Therefore, all trout
			      swim.
		\end{itemize}
	\item[issue] axioms that seem self evident, can actually be false
		\begin{itemize}
			\item ex. Einstein showed that 2 points don't necessarily determine
			      a line
		\end{itemize}
\end{description}

\subsubsection{Aristotle's ``science'' vs Modern Science}
\begin{description}
	\item[modern science] accept axioms because confirmed by observation and
		experiment (not because it seems self evident)
\end{description}
\begin{itemize}
	\item Modern science stems from observation (c.f. Aristotle's syllogisms)
	\item Good scientific theory successfully predicts and explains
	      observations
	\item 2 types of logic
	      \begin{enumdescription}
		      \item[inductive logic] particular observation $\to$ general laws
		      \item[deductive logic] general laws $\to$ specific laws $\to$
		      particular predictions
	      \end{enumdescription}
	\item n.b. Currently no science of inductive logic
	\item Scientific development during Renaissance
	      \begin{enumerate}
		      \item experimental method: observation and experiment
		      \item greater use of mathematical methods
	      \end{enumerate}
	\item Aristotle -- 1850: logic theory stagnant
	\item George Boole 1854: modern logic (first use of mathematical methods in
	      logic)
	      \begin{itemize}
		      \item Aristotle's Theory of Syllogism now obsolete
		      \item Boole's novel idea was applying algebra to logic
	      \end{itemize}
\end{itemize}


