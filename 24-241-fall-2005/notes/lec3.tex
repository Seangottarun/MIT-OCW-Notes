\section{Sentential Calculus Intro}
\begin{description}
	\item[logic] science of \textit{valid} argument
	\item[valid argument] argument in which it isn't possible for the
		premises to be true and the conclusion false.
\end{description}

\subsection{Formal language}
\begin{itemize}
	\item Study formal languages simpler than English but rich enough to have
	      many common features
	\item Determine which arguments in formal language valid
	\item Then translate English $\to$ formal language such that if translated
	      argument valid in formal language, the English argument must be
	      valid too
	\item Can show an English argument is valid by providing translation into
	      demonstrably valid argument in formal language
\end{itemize}

\begin{description}
	\item[sentenial calculus (SC)] deals with propositions and relations between
		propositions including construction of arguments based on them
		\begin{itemize}
			\item Models how compound English sentences are built from simple
			      English sentences
			\item Sentences in SC built up by \textit{atomic sentences} using
			      \textit{connectives} (equivalent to conjunctions FANBOYS)
		\end{itemize}
\end{description}

\subsubsection{Connectives}
\begin{enumdescription}
	\item[Logical AND ($\land$)] true if both true
	\[
		\begin{array}{|c c|c|}
			A & B & A \land B \\
			\hline
			T & T & T         \\
			T & F & F         \\
			F & T & F         \\
			F & F & F
		\end{array}
	\]
	\item[Logical OR ($\lor$)] true if at least one is true
	\[
		\begin{array}{|c c|c|}
			A & B & A \lor B \\
			\hline
			T & T & T        \\
			T & F & T        \\
			F & T & T        \\
			F & F & F
		\end{array}
	\]
	\item[Logical NOT ($\lnot$)] true if false
	\[
		\begin{array}{|c c|c|c|}
			A & B & \lnot A & \lnot B \\
			\hline
			T & T & F       & F       \\
			T & F & F       & T       \\
			F & T & T       & F       \\
			F & F & T       & T
		\end{array}
	\]
	\item[if $\dots$ then OR only if ($\to$)] true if both true or antecedent
	$A$ is false
	\[
		\begin{array}{|c c|c|}
			A & B & A \to B \\
			\hline
			T & T & T       \\
			T & F & F       \\
			F & T & T       \\
			F & F & T
		\end{array}
	\]
	\begin{itemize}
		\item n.b English \textit{if $\dots$ then} is
		      \[
			      \begin{array}{|c c|c|}
				      A & B & \text{if $A$, then $B$ } \\
				      \hline
				      T & T & T                        \\
				      T & F & F                        \\
				      F & T & \rule{1cm}{0.15mm}       \\
				      F & F & \rule{1cm}{0.15mm}
			      \end{array}
		      \]
		\item Logicians arbitrarily filled in blanks; \textit{if $\dots$
			      then} statements in logic considered true if antecedent $A$ doesn't
		      occur
	\end{itemize}
	\item[if and only if ($\leftrightarrow$)] true if both or neither are true
	(i.e. they match up)
	\[
		\begin{array}{|c c|c|}
			A & B & A \leftrightarrow B \\
			\hline
			T & T & T                   \\
			T & F & F                   \\
			F & T & F                   \\
			F & F & T
		\end{array}
	\]
\end{enumdescription}
\begin{itemize}
	\item ex. combinations of connectives
	      \[
		      \begin{array}{|c c|c|c|c|}
			      A & B & A \land \lnot B & \lnot (A \land B) & \lnot (A \lor B) \\
			      \hline
			      T & T & F               & F                 & F                \\
			      T & F & T               & T                 & F                \\
			      F & T & F               & T                 & F                \\
			      F & F & F               & T                 & T
		      \end{array}
	      \]
\end{itemize}
\subsubsection{English equivalent connectives}
\begin{itemize}
	\item See Table \ref{tab:eng_connectives} for English equivalents of logical
	      connectives.
	\item Here, the formal language sentence $A$ means the same as the English
	      sentence ``Jack went up the hill''
	\item Similarly, the formal language sentence B means the same as ``Jill went up
	      the hill''
\end{itemize}
\begin{table}[htbp]
	\centering
	\begin{tabular}{|c|c|c|}
		\hline
		Connective     & SC sentence           & English sentence                            \\
		\hline
		AND            & $A \land B$           & Jack and Jill went up the hill              \\
		\hline
		OR             & $A \lor B$            & Jack and/or Jill went up the hill           \\
		\hline
		NOT            & $\lnot A$             & Jack didn't go up the hill                  \\
		\hline
		NAND           & $\lnot(A \land B)$    & \shortstack{Jack and Jill didn't both go up
		the hill.                                                                            \\
		I.e. only one or none of them went up the hill.                                      \\
			By De Morgan's Rule
		$\lnot(A \land B) = (\lnot A) \lor (\lnot B)$}                                       \\
		\hline
		NOR            & $\lnot(A \lor B)$     & \shortstack{Neither Jack nor Jill went up
		the hill.                                                                            \\
		I.e. none of them went up the hill.                                                  \\
			By De Morgan's Rule $\lnot(A \lor
		B) = (\lnot A) \land (\lnot B)$)}                                                    \\
		\hline
		\shortstack{if $\dots$ then                                                          \\ (aka only if)} & $A \to B$             & \shortstack{If Jack went up the hill, then
		so did Jill.                                                                         \\
		Equivalently, Jack went up the hill only if Jill did                                 \\
			I.e. true if both go up the hill or if Jack doesn't go up the
		hill.}                                                                               \\
		\hline
		if and only if & $A \leftrightarrow B$ & \shortstack{Jack went up the hill if and
			only if Jill
		went up the hill.                                                                    \\
		I.e. True if both or neither of them went up the hill}                               \\
		\hline
	\end{tabular}
	\caption{Mapping between logical connectives and English equivalents}
	\label{tab:eng_connectives}
\end{table}

\subsection{SC terminology}
\begin{definition}[disjunction $\varphi \lor \psi$]
	A \textit{disjunction} is a sentence of the form $\varphi \lor \psi$ where
	$\varphi$ and $\psi$ are its \textit{disjuncts}.
\end{definition}
\begin{definition}[conjunction $\varphi \land \psi$]
	A \textit{conjunction} is a sentence of the form $\varphi \land \psi$ where
	$\varphi$ and $\psi$ are its \textit{conjuncts}.
\end{definition}
\begin{definition}[negation $\lnot \varphi$]
	A \textit{negation} is a sentence of the form $\lnot \varphi$ where
	$\varphi$ is its \textit{negatum}.
\end{definition}
\begin{definition}[conditional $\varphi \to \psi$]
	A \textit{conditional} is a sentence of the form $\varphi \to \psi$, where
	$\varphi$ is its \textit{antecedent} and $\psi$ is its \textit{consequent}
\end{definition}
\begin{definition}[biconditional $\varphi \leftrightarrow \psi$]
	A \textit{biconditional} is a sentence of the form $\varphi \leftrightarrow
		\psi$ where $\varphi$ and $\psi$ are its \textit{components}
\end{definition}
\begin{definition}[SC language]
	An SC \textit{language} is determined by specifying a nonempty set of
	simple expressions to serve as \textit{atomic sentences}. The sentences of
	the language, referred to as SC sentences, constitute the smallest class of
	expression that
	\begin{enumerate}
		\item contains the atomic sentences
		\item contains $\lnot \varphi$ wherever it contains $\varphi$
		\item contains $(\varphi \lor \psi), (\varphi \land \psi), (\varphi
			      \to \psi), \text{ and } (\varphi \leftrightarrow \psi)$
		      wherever it contains both $\varphi$ and $\psi$
	\end{enumerate}
\end{definition}

\begin{itemize}
	\item I.e. if a language \rule{1cm}{0.15mm} is the set of all sets of
	      expressions that satisfy the 3 conditions,
	      then an expression is an SC sentence $\iff$ it is a member of every
	      member
	      of \rule{1cm}{0.15mm}
	\item Plain English version of SC sentence definition
	      \begin{enumerate}
		      \item Every atomic sentence is an SC sentence
		      \item Result of writing $\lnot$ in \textit{in front of an SC
			            sentence} is \textit{always an SC sentence} ($\lnot SC$)
		      \item Conjunctions, disjunctions, conditionals,
		            biconditionals of SC
		            sentences is an SC sentence
		      \item Nothing is an SC sentence unless it is required to be
		            by 3 clauses above
	      \end{enumerate}
\end{itemize}

\begin{theorem}[Unique Readability]
	A sentence of an SC language built up from uniquely from sequential letters. I.e.
	use parentheses to prevent ambiguity.
\end{theorem}
\begin{itemize}
	\item English doesn't have unique readability (sometimes ambiguous)
\end{itemize}
\begin{proof}
	Can represent structure of any given sentence by a finite tree in which each
	node of the tree is labelled by a sentence. The given sentence is the root of
	the tree. Whenever a node is labelled by a conjunction, there will be 2 other
	nodes directly beneath the node, each labelled with one of the conjuncts. Same
	for disjunctions, conditionals, biconditionals. A negation node has just one
	node directly beneath it with the negatum.
	Leaves of trees labelled with atomic sentences.

	Unique readability is proved by showing that each sentence is associated
	with one and only one labeled tree
\end{proof}

\begin{example}[Tree for logical sentence $\lnot(A \land (B \land \lnot (\lnot A \land C)))$]
	\begin{center}
		% \Tree [.{$\lnot (A \land (B \land \lnot (\lnot A \land C)))$} ]
		\Tree [.{$\lnot (A \land (B \land \lnot (\lnot A \land C)))$} [
					.{$(A \land (B \land \lnot (\lnot A \land C)))$} A
						[.{$(B \land \lnot(\lnot A \land C))$} B
								[.{$\lnot (\lnot A \land C)$}
										[.{$(\lnot A \land C)$}
												[.{$\lnot A$} A ]
											C ] ]]]]
		% \Tree [.{$\lnot (A \land (B \land \lnot (\lnot A \land C)))$} [.{$R_*>1$} {$i_M\leq h_*$} ]  ]
	\end{center}
	% \begin{center}
	% 	\Tree[.IP [.NP [.Det \textit{the} ]
	% 					[.N\1 [.N \textit{package} ]]]
	% 			[.I\1 [.I \textsc{3sg.Pres} ]
	% 				[.VP [.V\1 [.V \textit{is} ]
	% 							[.AP [.Deg \textit{really} ]
	% 									[.A\1 [.A \textit{simple} ]
	% 										\qroof{\textit{to use}}.CP ]]]]]]
	% \end{center}
\end{example}

\begin{itemize}
	\item Because of unique readability, we know that every SC sentence fully
	      falls exactly into 1 of the following 6 categories
	      \begin{enumerate}
		      \item atomic sentence
		      \item conjunction
		      \item disjunction
		      \item conditional
		      \item biconditional
		      \item negation
	      \end{enumerate}
\end{itemize}
